% !TeX root =  main.tex

\chapter{Trigonometric Identities and Equations}

\section{Simplifying Trigonometric Expressions}

\begin{note}[Fundamental Identities]
  \begin{multicols}{4}
  Pythagorean identities
\begin{align*}
  \sin^2 x + \cos^2 x &= 1\\
  1 + \tan^2 x &= \sec^2 x\\
  1 + \cot^2 x &= \csc^2 x
\end{align*}

  \columnbreak

Even-odd identities
    \begin{align*}
      \sin(-x)  &= - \sin x\\
      \cos(-x)  &= \cos x\\
      \tan(-x)  &= -\tan x
    \end{align*}

    \columnbreak

    Reciprocal identities
    \begin{align*}
      \cot x &= \frac{1}{\tan x}\\
      \csc x &= \frac{1}{\sin x}\\
      \sec x &= \frac{1}{\cos x}
    \end{align*}
 
    \columnbreak

    Quotient identities 
    \begin{align*}
      \tan x & =\dfrac{\sin x}{\cos x}\\
      \cot x &=\dfrac{\cos x}{\sin x}
    \end{align*}
  
    % \columnbreak

    % % Cofunction identities

    % \begin{align*}
    %   \sin\left(\frac{\pi}{2} - x\right) &= \cos x\\
    %   \cos\left(\frac{\pi}{2} - x\right) &= \sin x\\
    %   \tan\left(\frac{\pi}{2} - x\right) &= \cot x
    % \end{align*}
  \end{multicols}
\end{note}

\begin{exercise}
  Verify the trigonometric identity.\\
  \begin{enumerate*}
    \item $\tan \theta \cos \theta=\sin \theta$
    \item $\dfrac{{\sec}^2 \theta-1}{{\sec}^2 \theta}={\sin}^2 \theta$\hfill\null
  \end{enumerate*}
\end{exercise}

\begin{exercise}
 Simplify the trigonometric identity.\\
  \begin{enumerate*}
    \item $\dfrac{{\sin}^2(-\theta)-{\cos}^2(-\theta)}{\sin(-\theta)-\cos(-\theta)}$
    \item $(1-{\cos}^2 x)(1+{\cot}^2 x)$\hfill\null
  \end{enumerate*}
\end{exercise}

\newpage
\section*{Exercises}

\begin{exercise}
  Verify the trigonometric identity.\\
  \begin{enumerate*}
    \item $\dfrac{\tan x}{\sec x}\sin (-x)=\cos ^2x$
    \item $\dfrac{\cos ^2 \theta -\sin ^2 \theta }{1-\tan ^2 \theta }=\sin ^2 \theta$
    \item $\dfrac{\sec (-x)}{\tan x+\cot x}=-\sin (-x)$\hfill\null
  \end{enumerate*}
\end{exercise}
\begin{exercise}
  Simplify the trigonometric identity.\\
  \begin{enumerate*}
    \item $\tan x\sin x+\sec x\cos^2x$
    \item $\dfrac{\cot t+\tan t}{\sec (-t)}$
    \item $\dfrac{1-\cos ^2 x}{\tan ^2 x}+2\sin ^2 x$\hfill\null
  \end{enumerate*}
\end{exercise}

\newpage

\section{Formulas of Angle Addition}
\begin{theorem}[Sine and Cosine of Sum or Difference of Angles\footnotemark]
\begin{align*}
\sin(\alpha+\beta) & =\sin \alpha \cos \beta + \sin \alpha \cos \beta\\
\cos(\alpha+\beta) & =\cos \alpha \cos \beta-\sin \alpha \sin \beta\\
\sin(\alpha-\beta) &=\sin \alpha \cos \beta -  \sin \alpha \cos \beta\\
\cos(\alpha-\beta) &=\cos \alpha \cos \beta+\sin \alpha \sin \beta
\end{align*}  
\end{theorem}

\footnotetext{One proof is to use Euler formula: $e^{i\alpha}=\cos\alpha + i\sin\alpha$.}

\begin{example}
  Find the exact value.\\
  \begin{enumerate*}
    \item $\cos(75\degree)$
    \item $\sin(135\degree)$\hfill\null
  \end{enumerate*}
\end{example}


\begin{example}
  Find the exact value of $\sin\left ({\cos}^{-1}\left(\frac{1}{2}\right)+{\sin}^{-1}\left(\frac{3}{5}\right)\right)$.
\end{example}

\newpage

\begin{example}
  Given $\sin \alpha=\frac{3}{5}, \quad 0<\alpha<\frac{\pi}{2}$, and $\cos \beta=-\frac{5}{13}, \quad \pi<\beta<\frac{3\pi}{2}$, find\\
  \begin{enumerate*}
    \item $\sin(\alpha+\beta)$
    \item $\cos(\alpha-\beta)$
    \item $\tan(\alpha+\beta)$
    \item $\csc(\alpha-\beta)$\hfill\null
  \end{enumerate*}
\end{example}


\begin{example}
  Prove the following identities using the identities for sum/difference of angles.
  \begin{enumerate*}
    \item $\sin(\frac{\pi}{2}-x)=\cos x$
    \item $\sin(\pi - x)=\sin x$
    \item $\cos(\pi - x)=-\cos x$\hfill\null
  \end{enumerate*}
\end{example}

\newpage

\begin{example}
  Verify the identity $\sin(\alpha+\beta)+\sin(\alpha-\beta)=2\sin \alpha \cos \beta$.
\end{example}

\begin{example}
  Verify the identity $\cos(2\alpha)=\cos^2\alpha-\sin^2\alpha$.
\end{example}

\newpage
\section*{Exercises}
\begin{exercise}
  Find the exact value.\\
  \begin{enumerate*}
    \item $\cos(75\degree)$
    \item $\sin(135\degree)$\hfill\null
  \end{enumerate*}
\end{exercise}

\begin{exercise}
  Find the exact value of $\cos\left(\cos^{-1}\left(\frac{1}{3}\right) - \sin^{-1}\left(\frac45\right)\right)$.
\end{exercise}

\begin{exercise}
  Given $\sin \alpha=-\frac{4}{5}, \quad \pi<\alpha<\frac{3\pi}{2}$, and $\cos \beta=\frac{12}{13}, \quad 0<\beta<\frac{\pi}{2}$, find\\
  \begin{enumerate*}
    \item $\sin(\alpha-\beta)$
    \item $\cos(\alpha+\beta)$
    \item $\cot(\alpha-\beta)$\hfill\null
  \end{enumerate*}
\end{exercise}

\newpage

\begin{exercise}
  Prove the following identities using the identities for sum/difference of angles.
  \begin{enumerate*}
    \item $\cos(x+\frac{\pi}{2})=\sin x$
    \item $\sin(x-\pi)=-\sin x$
    \item $\cos(x+\pi)=-\cos x$\hfill\null
  \end{enumerate*}
\end{exercise}

\begin{exercise}
  Verify the identity $\cos(\alpha-\beta)-\cos(\alpha+\beta)=2\sin \alpha \sin \beta$.
\end{exercise}

\begin{exercise}
  Verify the identity $\sin(2\alpha)=2\sin \alpha \cos \alpha$.
\end{exercise}


\newpage

\section{Double and Half Angle Formulas}

\begin{corollary}[Double Angle Identities]
  \begin{align*}
    \sin(2\alpha)=&2\sin\alpha\cos\alpha\\
    \cos(2\alpha)=&\cos^2\alpha-\sin^2\alpha\\
    =&2\cos^2\alpha -1\\
    =&1-2\sin^2\alpha
  \end{align*}
\end{corollary}

\begin{corollary}[Half Angle Identities]
  \begin{align*}
    \sin^2\left(\dfrac{\theta}{2}\right)=&\dfrac{1-\cos\theta}{2}\\
    \cos^2\left(\dfrac{\theta}{2}\right)=&\dfrac{1+\cos\theta}{2}
  \end{align*}
\end{corollary}

\begin{example}
  Find the exact value.\\
  \begin{enumerate*}
    \item $\sin\left(2\cos^{-1}\left(\dfrac{3}{5}\right)\right)$
    \item $\tan\left(2\sin^{-1}\left(\dfrac{3}{5}\right)\right)$\hfill\null
  \end{enumerate*}
\end{example}

\newpage

\begin{example}
  Verify the identity ${\cos}^4\theta-{\sin}^4 \theta=\cos(2\theta)$.
\end{example}

\begin{example}
  Verify the identity:  $\tan(2\theta)=2\cot\theta-\tan\theta$
\end{example}

\begin{example}
  Write an equivalent expression for $\cos^4 x$ that does not involve any powers of sine or cosine greater than $1$.
\end{example}

\newpage

\begin{example}
  Find $\sin 15\degree$ and $\cos 15\degree$.
\end{example}

\begin{example}
  Given that $\tan\alpha=\dfrac{8}{15}$ and $\alpha$ lies in quadrant III, find the exact value of the following:\\
\begin{enumerate*}
  \item $\sin\left(\dfrac{\alpha}{2}\right)$
  \item $\cos\left(\dfrac{\alpha}{2}\right)$
  \item  $\tan\left(\dfrac{\alpha}{2}\right)$
\end{enumerate*}
\end{example}

\newpage

\section*{Exercises}

\begin{exercise}
  Find the exact value.\\
  \begin{enumerate*}
    \item $\cos\left(2\sin^{-1}\left(\dfrac{4}{5}\right)\right)$
    \item $\tan\left(2\cos^{-1}\left(\dfrac{4}{5}\right)\right)$
  \end{enumerate*}
\end{exercise}

\begin{exercise}
  Verify the identity $(\cos\theta-\sin\theta)^2=1-\sin(2\theta)$.
\end{exercise}

\newpage

\begin{exercise}
  Write an equivalent expression for $\sin^4 x$ that does not involve any powers of sine or cosine greater than $1$.
\end{exercise}

\begin{exercise}
  Given that $\sin \alpha=-\dfrac{4}{5}$ and $\alpha$ lies in quadrant IV, find the exact value of $\tan\left(\dfrac{\alpha}{2}\right)$.
\end{exercise}

\newpage

\section{Sum-to-Product and Product-to-Sum Formulas}

\begin{corollary}[The Product-to-Sum Formulas]
\begin{align*}
  \cos \alpha \cos \beta =& \dfrac{1}{2}[\cos(\alpha - \beta) + \cos(\alpha + \beta)]\\
  \sin \alpha \cos \beta =& \dfrac{1}{2}[\sin(\alpha+\beta)+\sin(\alpha - \beta)]\\
  \sin \alpha \sin \beta =& \dfrac{1}{2}[\cos(\alpha-\beta)-\cos(\alpha+\beta)]
\end{align*}
\end{corollary}

\begin{corollary}[The Sum-to-Produc Formulas]
\begin{align*}
  \cos\alpha + \cos\beta = & 2\cos\left(\dfrac{\alpha+\beta}{2}\right)\cos\left(\dfrac{\alpha-\beta}{2}\right)\\
  \sin\alpha + \sin\beta = & 2\sin\left(\dfrac{\alpha+\beta}{2}\right)\cos\left(\dfrac{\alpha-\beta}{2}\right)\\
  \cos\alpha - \cos\beta = & 2\sin\left(\dfrac{\alpha+\beta}{2}\right)\sin\left(\dfrac{\alpha-\beta}{2}\right)\\
\end{align*}
\end{corollary}

\begin{example}
  Write the following product as a sum\\
  \begin{enumerate*}
    \item $2\cos\left(\dfrac{7x}{2}\right) \cos\left(\dfrac{3x}{2}\right)$
    \item $\sin\left(3\theta\right) \cos\left(5\theta\right)$\hfill\null
  \end{enumerate*}
\end{example}


\begin{example}
  Write the following difference or sum expression as a product.
  \begin{enumerate*}
    \item $\sin(3\theta)-\sin\theta$
    \item $\cos(2\theta)+\cos(4\theta)$
    \item $\cos(4\theta)-\cos(2\theta)$
    \item $\sin\theta-\cos\theta$
    \hfill\null
  \end{enumerate*}
\end{example}

\newpage

\begin{example}
  Evaluate
  \begin{enumerate*}
    \item $\cos(15\degree)-\cos(75\degree)$
    \item $\sin(15\degree)+\sin(45\degree)$
    \item $\sin(45\degree)-\cos(135\degree)$
    \hfill\null
  \end{enumerate*}
\end{example}

\begin{example}
  Prove the identity:
$$
\dfrac{\cos(4t)-\cos(2t)}{\sin(4t)+\sin(2t)}=-\tan t
$$
\end{example}

\newpage
\section*{Exercises}

\begin{exercise}
  Write the following product as a sum\\
  \begin{enumerate*}
    \item $\sin\left(\dfrac{\theta}{2}\right) \cos\left(\dfrac{5\theta}{2}\right)$
    \item $\sin\left(4\theta\right) \sin\left(2\theta\right)$\hfill\null
  \end{enumerate*}
\end{exercise}

\begin{exercise}
  Write the following difference or sum expression as a product.
  \begin{enumerate*}
    \item $\sin(5\theta)-\sin\theta$
    \item $\cos(\theta)+\sin(\theta)$
    \item $\cos(3\theta)+\cos(5\theta)$\hfill\null
  \end{enumerate*}
\end{exercise}

\newpage

\begin{exercise}
  Evaluate
  \begin{enumerate*}
    \item $\sin(75\degree)-\cos(75\degree)$
    \item $\sin(45\degree)+\sin(135\degree)$
  \end{enumerate*}
\end{exercise}

\begin{exercise}
  Prove the identity $\sin x + \sin(3x)=4\sin x\cos^2 x$.
\end{exercise}

\newpage

\section{Solving Trigonometric Equations}
\begin{example}
  Find all possible exact solutions for the equation.\\
  \begin{enumerate*}
    \item $\cos \theta=\dfrac{1}{2}$
    \item $\sin \theta=\dfrac{1}{2}$
    \item $\tan \theta=\dfrac{\sqrt{3}}{3}$\hfill\null
  \end{enumerate*}
\end{example}

\begin{example}
  Solve the equation exactly: $2\cos \theta-3=-5$, $0\le \theta<2\pi$.
\end{example}

\begin{example}
  Solve the equation exactly:  $2\sin^2\theta-1=0$, $0\le \theta<2\pi$.
\end{example}

\newpage

\begin{example}
  Solve the equation exactly: $\cos^2\theta+3 \cos \theta-1=0$, $0\le \theta<2\pi$.
\end{example}

\begin{example}
  Solve the equation exactly over the interval $0\le x<2\pi$
\[\cos x \cos(2x)+\sin x \sin(2x)=\dfrac{\sqrt{3}}{2}.\]
\end{example}

\begin{example}
  Solve the equation exactly: $\cos(2\theta)=\cos\theta$.
\end{example}

\newpage

\begin{example}
  Solve the equation exactly: $2\cos^2\theta-3\sin\theta=3$.
\end{example}

\begin{example}
  Solve the equation quadratic in form exactly: $2 {\sin}^2 \theta-3 \sin \theta+1=0$, $0\le\theta<2\pi$
\end{example}

\newpage
\section*{Exercises}

\begin{exercise}
  Solve the equation exactly: $2\sin \theta\cos\theta-\sqrt{3}=0$, $0\le \theta<2\pi$.
\end{exercise}

\begin{exercise}
  Solve the equation exactly: $\cos^2\theta+3 \cos \theta-1=0$, $0\le \theta<2\pi$.
\end{exercise}

\begin{exercise}
  Solve the equation quadratic in form exactly: $2 {\sin}^2 \theta-3 \sin \theta+1=0$, $0\le\theta<2\pi$
\end{exercise}

\newpage

\begin{exercise}
  Solve the equation exactly over the interval $0\le x<2\pi$
\[\sin x \cos(2x)+\cos x \sin(2x)=\dfrac{1}{2}.\]
\end{exercise}

\begin{exercise}
  Solve the equation exactly: $sin(2\theta)=cos\theta$.
\end{exercise}

\begin{exercise}
  Solve the equation exactly: $2\sin^2\theta=3\cos\theta-3$.
\end{exercise}
